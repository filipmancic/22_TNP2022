 \documentclass[12pt,twoside]{article}
\usepackage[utf8]{inputenc}
\usepackage[margin=1in,top=2.8cm,headheight=17pt]{geometry}
\usepackage{graphicx}
\usepackage{multirow}
\graphicspath{ {./images/} }
\usepackage{todonotes}
\usepackage{fancyhdr}
\pagestyle{fancy}
\fancyhead[LE, RO]{\thepage}
\fancyhead[RE]{\thechapter.\ \ \chaptername}
\fancyhead[LO]{\nouppercase{\rightmark}}
\fancyfoot{}
% za cirilicu
\usepackage[utf8]{inputenc}
% \usepackage{cmsrb} % namesti ovo kasnije da ti radi
\usepackage[OT2]{fontenc}
\usepackage[serbian]{babel}
% srpski navodnici
\def\zn{,\kern-0.09em,}
\def\zng{'\kern-0.09em'}
% forsiraj cirilicu
\newcommand{\cirilica}[1]{{\fontencoding{OT2}{\selectfont{#1}}}}
\title{Obrazovanje u Srbiji}
\author{Marina Vracharic1\and Anastasija Mandic1\and Jovana Tasovac\and Filip Manchic1}
\usepackage{titlesec}
\titleformat{\chapter}[hang]{\Huge\bfseries}{\thechapter{\ \ }}{0pt}{\Huge\bfseries}
\usepackage[hidelinks]{hyperref}
\begin{document}
\maketitle
    \renewcommand\contentsname{Sadrzhaj}
    \tableofcontents{}
    \thispagestyle{empty}
    \clearpage
    \thispagestyle{empty}
     \cleardoublepage
    

    % literatura
    \renewcommand\bibname{\cirilica{Literatura}}
    \nocite{*}
    
    
  \section{Obrazovanje kroz istoriju}
  Средњовековна српска држава је располагала одређеним школским системом. Како се уздизала и јачала све више је растао број писмених и образованих људи који су јој били потребни. У побожној православној Србији, цркве и манастири бивали су све бројнији, а посебно од времена Светог Саве, били су расадници писмености и школовања деце. Али нису били једини. На српском двору и у кућама великих властелина, деца су се учила разним вештинама: писању, читању, украшавању, ткању, трговини, занатима…све до борилачких вештина.
  \\Veliki znachaj za srpsko obrazovanje kroz istoriju nosi Resavska shkola prepisivacha. Ресавска школа постојала је током прве половине 15. века и радила је у оквиру манастира Манасије, који је подигао деспот Стефан Лазаревић, као своју задужбину. У њој су се окупљали учени монаси писци, преводиоци, књижевници, преписивачи који су украшавали рукописе и књиге, због чега је манастир представљао симбол духовности и просвећености током неколико наредних векова. Један од главних сарадника Ресавске школе био је Константин Филоsоф. По правилима ове школе радило се у Љубостињи, Хиландару, Пећкој патријаршији, Дечанима, а утицај се осећао и у Македонији, Бугарској, Румунији и Русији. \\Упркос значају и утицајима Ресавске школе, данас је нажалост у друштву раширен израз „ресавска школа“, који има потпуно другачије значење.\\ Преписи и преводи Ресавске школе сматрају се узорнима, по којима се до 18. века мерио квалитет рукописа.\\ Захваљујући раду и утицају Ресавске школе српски народ је сачувао свој језик. \\ У време турских освајања манастир Ресава делио је судбину читаве српске Деспотовине. Књижевна активност манастира ометена је најездом Турака и пропашћу деспотовине. Један део калуђера повукао се са последњим Бранковићима у Срем и наставио је ту акцију у фрушкогорским манастирима (нпр. у Крушедолу), други је отишао у Влашку, а затим у Русију, поневши са собом правописна правила и списе Константина Философа, док се најмањи део окупио у Јужној Србији, у Пећкој патријаршије и Дечанима.\\
  \\ Падом Србије под турско ропство, све се изменило. За генерацију или две нестало је српске властеле, читавог расадника писмености и учености. Оно што је код Срба било прогресивно, Турци су свесно и систематски гушили и уништавали. \\После пада српске средњовековне државе српски народ се није развијао на једном заједничком простору и под истим политичким, верским и културним утицајима.\\
  \\У 18. и 19. веку већина српског народа живи под турском и аустроугарском влашћу. Развитак школства код Срба у Хабзбуршкој монархији у 18. веку одвијао се у неколико карактеристичних етапа:\\\\
  \begin{enumerate}
      \item Sрпско-словенска етапа од 1690-1726. године;\\ Школе вероисповедног обележја почињу се оснивати одмах након доласка Срба у Аустрију. Школе су носиле назив "тривијалне школе" јер су биле организоване по узору на сличне грчке школе у Цариградској патријаршији
      \item Rуско словенска етапа од 1726-1749. године
      \item Vреме митрополита Павла Ненадовића 1749-1768 године
      \item Pериод просвећености који је био праћен терзијанским и јозефинским реформама.
  \end{enumerate}
  \\\\
  У Београду је између 1718. и 1739. постојала Мала српска школа. Почетком 18. века, у порти цркве Светог Ђорђа, 1703. год. основана је прва школа у Петроварадинском Шанцу ( Новом Саду) - Српска православна основна школа. Аврам Мразовић је 1. маја 1778. у Сомбору основао српску основну школу „Норму“, која је најстарија установа за образовање учитеља за словенско становништво на југу Европе. Карловачки митрополит Стеван Стратимировић је 1791. у Сремским Карловцима основао Карловачку гимназију. Током Првог српског устанка, у Београду је 1808. основана Велика школа заслугом Ивана Југовића и Доситеја Обрадовића. У Крагујевцу је октобра 1838. основан Лицеум Књажества сербског, прва виша школа у Србији. Он је 1841. премештен у Београд. Лицеј је постојао до 1863. када прераста у Велику школу, која је имала три факултета: Филозофски, Технички и Правни.\\\\Београдски универзитет је основан 1905. После Другог све{т}sког рата из Београдског универзитета су се издвојили Универзитет у Новом Саду (1960), Нишу (1965), Приштини (1970), Подгорици (1974) и Крагујевцу (1976).\\До примене Болоњске декларације, 2006, и свеобухватних измена образовног система, Србија је примењивала систем наслеђен из социјалистичке Југославије. Предшколско образовање било је изборно, а основно и средње образовање идентични садашњим. Применом Болоњске декларације школске 2005/06. године, бивша Диплома високог образовања је изједначена са мастером, а магистратура са прве две године докторских студија (још једна година до доктората), обоје услед једнаких година учења. Постдипломско образовање је обухватало оно што данас обухватају други и трећи део високог — магистратуру и докторат.\\
  Са образовањем деце се почиње у предшколским установама које се одржавају у локалном вртићу, те је оно први део образовања. Оно је обавезно од школске 2006/07. године. Траје бар 4 сата дневно бар 6 месеци у години уписа у први разред, за децу од пет или шест година, а циљ му је упознавање ученика са образовним системом и припрема за основну школу.\\
  Деца се уписују у основну школу са шест или седам година. Као и предшколско, и основно образовање је обавезно. Основна школа траје осам година и подељена је на два периода:
\begin{enumerate}
    \item први циклус основног образовања (од 1. до 4. разреда)
    \item други циклус основног образовања (од 5. до 8. разреда)
\end{enumerate}

\section{Reference}
\begin{enumerate}
\item \url{https://sr.wikipedia.org/sr-el/%D0%9E%D0%B1%D1%80%D0%B0%D0%B7%D0%BE%D0%B2%D0%B0%D1%9A%D0%B5_%D1%83_%D0%A1%D1%80%D0%B1%D0%B8%D1%98%D0%B8}
\item \url{https://www.bbc.com/serbian/lat/srbija-59409115}
\item \url{https://scindeks-clanci.ceon.rs/data/pdf/0353-7129/2008/0353-71290803173S.pdf}
\item \url{https://okc.rs/sta-jeste-a-sta-nije-onlajn-nastava/}
\item \url{https://www.stat.gov.rs/sr-latn/oblasti/obrazovanje/}
\item \url{https://www.ozonpress.net/drustvo/danas-je-medjunarodni-dan-pismenosti-statistika-u-srbiji-porazavajuca/}
\end{enumerate}
\end{document}


